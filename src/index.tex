\begin{center}
    \Large
    \textbf{Instrumentación Computacional Inteligente}
    \textbf{Reconstrucción de datos de estaciones meteorológicas}

    \vspace{0.3cm}
    \Large
    Fernando Ayats Llamas
    \vspace{0.3cm}
    2023-05-25
\end{center}

\tableofcontents

\section{Abstracto}

En este trabajo se estudia la reconstrucción de datos de temperatura recogidos
por dos estaciones. Las estaciones en cuestión son las localizadas en Río San
Pedro (RSP) y Embalse de Charco Redondo (ECR). Los datos analizados son unas
series temporales simples, con los datos de temperatura en grados centígrados,
durante un período de tiempo de una semana.

Para ello, se usado el lenguaje de programación Julia, por su facilidad de uso
para tareas de limpieza y análisis de datos. En este documento se presenta el
análisis de Río San Pedro, derivado de un cuaderno de Jupyter. El código fuente
del análisis, al igual que el código fuente de este documento se encuentran en
mi repositorio en \href{https://github.com/viperML/ici-reconstruct}{github.com/viperML/ici-reconstruct}. Como el
proceso para el análisis de la otra estación es trivial una vez ya se ha
realizado uno, queda el notebook de ECR en el repositorio.

\include{render.tex}